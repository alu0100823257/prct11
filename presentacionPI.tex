\documentclass{beamer}
\usepackage[utf8]{inputenc}
\usepackage{graphicx}

\newtheorem{definicion}{Definición}
\newtheorem{ejemplo}{Ejemplo}

%%%%%%%%%%%%%%%%%%%%%%%%%%%%%%%%%%%%%%%%%%%%%%%%%%%%%%%%%%%%%%%%%%%%%%%%%%%%%%%
\title[E numero PI]{Presentación historica}
\author[Shaila Verona Rodriguez]{Técnicas Experimentales}
\date[25-04-2014]{Viernes 25 de Abril de 2014}
%%%%%%%%%%%%%%%%%%%%%%%%%%%%%%%%%%%%%%%%%%%%%%%%%%%%%%%%%%%%%%%%%%%%%%%%%%%%%%%

%\usetheme{Madrid}
%\usetheme{Antibes}
%\usetheme{tree}
%\usetheme{classic}

%%%%%%%%%%%%%%%%%%%%%%%%%%%%%%%%%%%%%%%%%%%%%%%%%%%%%%%%%%%%%%%%%%%%%%%%%%%%%%%
\begin{document}
  
%++++++++++++++++++++++++++++++++++++++++++++++++++++++++++++++++++++++++++++++  
\begin{frame}

%  \includegraphics[width=0.15\textwidth]{img/ullesc}
%  \hspace*{7.0cm}
%  \includegraphics[width=0.16\textwidth]{img/fmatesc}
  \titlepage

  \begin{small}
    \begin{center}
     Facultad de Matemáticas \\
     Universidad de La Laguna
    \end{center}
  \end{small}

\end{frame}
%++++++++++++++++++++++++++++++++++++++++++++++++++++++++++++++++++++++++++++++  

%++++++++++++++++++++++++++++++++++++++++++++++++++++++++++++++++++++++++++++++  
\begin{frame}
  \frametitle{Índice}  
  \tableofcontents[pausesections]
\end{frame}
%++++++++++++++++++++++++++++++++++++++++++++++++++++++++++++++++++++++++++++++  


\section{Primera Sección}


%++++++++++++++++++++++++++++++++++++++++++++++++++++++++++++++++++++++++++++++  
\begin{frame}

\frametitle{Primera Sección}

\begin{definicion}
$\pi$  es la relación entre la longitud de una circunferencia y su diámetro, en geometría euclidiana. Se emplea frecuentemente en matemáticas, física e ingeniería. El valor numérico de $\pi$,es el siguiente:

     $\pi= 3,14159265358979323846$

El valor de $\pi$ se ha obtenido con diversas aproximaciones a lo largo de la historia, es una de las constantes matemáticas que más aparece. 
\alert{Matemáticas}~\cite{pi}
\end{definicion}

\end{frame}
%++++++++++++3,14159265359++++++++++++++++++++++++++++++++++++++++++++++++++++++++++++++++++  

\section{Segunda Sección}

%++++++++++++++++++++++++++++++++++++++++++++++++++++++++++++++++++++++++++++++  
\begin{frame}

\frametitle{Segunda Sección}

\begin{block}{Historia de PI}
  \begin{definicion}
 La búsqueda del mayor número de decimales del número $\Pi$ ha supuesto un esfuerzo constante de numerosos científicos a lo largo de la historia.
 Como por ejemplo en Egipto que afirmaban que el área de un círculo es similar a la de un cuadrado cuyo lado es igual al diámetro de un círculo disminuido en 1/9.
  \end{definicion}
\end{block}

\end{frame}
%++++++++++++++++++++++++++++++++++++++++++++++++++++++++++++++++++++++++++++++  

\section{Fórmulas}

\subsection{Algunas formulas matemáticas}
%++++++++++++++++++++++++++++++++++++++++++++++++++++++++++++++++++++++++++++++  
\begin{frame}

\frametitle{Primera Sección}

\begin{definicion}
$\pi$  es la relación entre la longitud de una circunferencia y su diámetro, en geometría euclidiana. Se emplea frecuentemente en matemáticas, física e ingeniería. El valor numérico de $\pi$,es el siguiente:
\begin{equation}
     $\pi= 3,14159265358979323846$
\end{equation}
El valor de $\pi$ se ha obtenido con diversas aproximaciones a lo largo de la historia, es una de las constantes matemáticas que más aparece. 
\alert{wikipedia}~\cite{wikipedia}
\end{definicion}

\end{frame}
\frametitle{Fórmula de Mesopotamia(aproximada)}

    $\pi 3 + (1)(8) = 3,125$

\end{frame}
%++++++++++++++++++++++++++++++++++++++++++++++++++++++++++++++++++++++++++++++ 
\begin{frame}
 \frametitle{Fórmula China (aproximación)}
   $\pi(22)/(7) = (355)/(113)$ 
\end{frame}


%++++++++++++++++++++++++++++++++++++++++++++++++++++++++++++++++++++++++++++++  
\begin{frame}
\frametitle{Fórmula Antigüedad clásica}


    $\pi(377)/(120)= 3.416$

\end{frame}


%++++++++++++++++++++++++++++++++++++++++++++++++++++++++++++++++++++++++++++++ 
\begin{frame}
 \frametitle{Fórmula de la India (solo 11 digitos)}
 
 $\pi=3,14159265359$
\end{frame}

%++++++++++++++++++++++++++++++++++++++++++++++++++++++++++++++++++++++++++++++  

\section{Bibliografía}
%++++++++++++++++++++++++++++++++++++++++++++++++++++++++++++++++++++++++++++++  
\begin{frame}
  \frametitle{Bibliografía}

  \begin{thebibliography}{10}

    \beamertemplatebookbibitems
    \bibitem[Matemáticas]{pi}  
Matematicas

    \beamertemplatebookbibitems
    \bibitem[wikipedia]{Wikipedia}  
    wikipedia
    {\small $http://es.wikipedia.org/wiki/$

    \beamertemplatebookbibitems
    \bibitem[URL: CTAN]{latex} 
    CTAN. {\small $http://www.ctan.org/$}

  \end{thebibliography}
\end{frame}

%++++++++++++++++++++++++++++++++++++++++++++++++++++++++++++++++++++++++++++++  
\end{document}
